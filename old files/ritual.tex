. BLESSING OF EPIPHANY WATER
on the Eve of Epiphany
(Approved by the Congregation of Sacred Rites, Dec. 6, 1890)
{This blessing comes from the Orient, where the Church has long emphasized in her celebration of Epiphany the mystery of our Lords baptism, and by analogy our baptism. This aspect is not neglected in western Christendom, although in practice we have concentrated on the visit of the Magi. Many years before the Latin Rite officially adopted the blessing of Epiphany water, diocesan rituals, notably in lower Italy, had contained such a blessing.}
1. At the appointed time the celebrant, vested in white cope (if a bishop, the mitre is worn but removed during the prayers), and the deacon and subdeacon, vested in white dalmatic and tunic respectively, come before the altar. They are preceded by acolytes, who carry the processional cross and lighted candles (which are put in their proper place), and by the other clergy. A vessel of water and a container of salt are in readiness in the sanctuary.
First the Litany of the Saints is sung, during which time all kneel. After the invocation "That you grant eternal rest," etc. the celebrant rises and sings the following two invocations, the second in a higher key:
That you bless this water. R. We beg you to hear us. That you bless and sanctify this water R. We beg you to hear us.
Then the chanters continue the litany up to and including the last Lord, have mercy.
After this the celebrant chants Our Father the rest inaudibly until:
P: And lead us not into temptation. 
All: But deliver us from evil.
2. Then the following psalms are sung:
Psalm 28
(for this psalm see Rite for Baptism of Adults)
Psalm 45
P: God is our refuge and our strength, * an ever-present help in distress.
All: Therefore we fear not, though the earth be shaken and mountains plunge into the depths of the sea;
P: Though its waters rage and foam * and the mountains quake at its surging.
All: The Lord of hosts is with us; * our stronghold is the God of Jacob.
P: There is a stream whose runlets gladden the city of God, * the holy dwelling of the Most High.
All: God is in its midst; it shall not be disturbed; * God will help it at the break of dawn.
P: Though nations are in turmoil, kingdoms totter, * His voice resounds, the earth melts away;
All: The Lord of hosts is with us; * our stronghold is the God of Jacob.
P: Come, see the deeds of the Lord, * the astounding things He has wrought on earth.
All: He has stopped wars to the end of the earth; * the bow he breaks; He splinters the spears; He burns the shields with fire.
P: Desist, and confess that I am God, * exalted among the nations, exalted upon the earth.
All: The Lord of hosts is with us; * our stronghold is the God of Jacob.
P: Glory be to the Father.
All: As it was in the beginning.
Psalm 146
P: Praise the Lord, for He is good; * sing praise to our God, for He is gracious; it is fitting to praise Him.
All: The Lord rebuilds Jerusalem; * the dispersed of Israel He gathers.
P: He heals the brokenhearted * and binds up their wounds.
All: He tells the number of the stars; * He calls each by name.
P: Great is our Lord and mighty in power; * to His wisdom there is no limit.
All: The Lord sustains the lowly; * the wicked He casts to the ground.
P: Sing to the Lord with thanksgiving; * sing praise with the harp to our God.
All: Who covers the heavens with clouds, * who provides rain for the earth;
P: Who makes grass sprout on the mountains * and herbs for the service of men;
All: Who gives food to the cattle, * and to the young ravens when they cry to Him.
P: He delights not in the strength of the steed, * nor is He pleased with the fleetness of men.
All: The Lord is pleased with those who fear Him, * with those who hope for His kindness.
P: Glory be to the Father.
All: As it was in the beginning.
The celebrant then chants:
Exorcism against Satan and the apostate angels
In the name of our Lord Jesus Christ and by His power, we cast you out, every unclean spirit, every devilish power, every assault of the infernal adversary, every legion, every diabolical group and sect; begone and stay far from the Church of God, from all who are made in the image of God and redeemed by the precious blood of the divine Lamb. Never again dare, you cunning serpent, to deceive the human race, to persecute the Church of God, nor to strike the chosen of God and to sift them as wheat. For it is the Most High God who commands you, He to whom you heretofore in your great pride considered yourself equal; He who desires that all men might be saved and come to the knowledge of truth. God the Father commands you. God the Son commands you. God the Holy Spirit commands you. The majesty of Christ, the eternal Word of God made flesh commands you; He who for the salvation of our race, the race that was lost through your envy, humbled Himself and became obedient even unto death; He who built His Church upon a solid rock, and proclaimed that the gates of hell should never prevail against her, and that He would remain with her all days, even to the end of the world. The sacred mystery of the cross commands you, as well as the power of all the mysteries of Christian faith. The exalted Virgin Mary, Mother of God commands you, who in her lowliness crushed your proud head from the first moment of her Immaculate Conception. The faith of the holy apostles Peter and Paul and the other apostles commands you. The blood of the martyrs and the devout intercession of all holy men and women commands you.
Therefore, accursed dragon and every diabolical legion, we adjure you by the living God, by the true God, by the holy God, by the God who so loved the world that He gave His only-begotten Son, that whoever believes in Him shall not perish but shall have life everlasting; cease your deception of the human race and your giving them to drink of the poison of everlasting damnation; desist from harming the Church and fettering her freedom. Begone Satan, you father and teacher of lies and enemy of mankind. Give place to Christ in whom you found none of your works; give place to the one, holy, Catholic, and apostolic Church, which Christ Himself purchased with His blood. May you be brought low under Gods mighty hand. May you tremble and flee as we call upon the holy and awesome name of Jesus, before whom hell quakes, and to whom the virtues, powers, and dominations are subject; whom the cherubim and seraphim praise with unwearied voices, saying: Holy, holy, holy, Lord God of hosts!
Next the choir sings the following antiphon and canticle:
Antiphon
Today the Church is espoused to her heavenly bridegroom, for Christ washes her sins in the Jordan; the Magi hasten with gifts to the regal nuptials; and the guests are gladdened with water made wine, alleluia.
Canticle of Zachary
Luke 1.68-79
P: "Blessed be the Lord, the God of Israel! * He has visited His people and brought about its redemption.
All: He has raised for us a stronghold of salvation * in the house of David His servant,
P: And redeemed the promise He had made * through the mouth of His holy prophets of old--
All: To grant salvation from our foes * and from the hand of all that hate us;
P: To deal in mercy with our fathers * and be mindful of His holy covenant,
All: Of the oath he had sworn to our father Abraham, * that He would enable us--
P: Rescued from the clutches of our foes--* to worship Him without fear,
All: In holiness and observance of the Law, * in His presence, all our days.
P: And you, my little one, will be hailed Prophet of the Most High; * for the Lords precursor you will be to prepare His ways;
All: You are to impart to His people knowledge of salvation * through forgiveness of their sins.
P: Thanks be to the merciful heart of our God! * a dawning Light from on high will visit us
All: To shine upon those who sit in darkness and in the shadowland of death, * and guide our feet into the path of peace."
P: Glory be to the Father.
All: As it was in the beginning.
Or instead of the "Benedictus" the "Magnificat" may be chosen (for the Magnificat see Blessing of Homes). At the end of the canticle the antiphon given above is repeated. Then the celebrant sings: 
P: The Lord be with you. 
All: May He also be with you.
Let us pray.
God, who on this day revealed your only-begotten Son to all nations by the guidance of a star, grant that we who now know you by faith may finally behold you in your heavenly majesty; through Christ our Lord. 
All: Amen.
Next he blesses the water: 
P: Our help is in the name of the Lord. 
All: Who made heaven and earth.
From here on the exorcism of salt and the prayer that follows it, the exorcism of water and the two prayers that follow it, the mixing of the salt and water and then the concluding prayer--all of these are the same as the ones used on pp. 395-97.
At the end of the blessing the priest sprinkles the people with the blessed water.
Lastly the "Te Deum" is sung (for the "Te Deum" and its oration see Renewal of the Marriage Vows).
